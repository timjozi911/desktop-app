%%%%%%%%%%%%%%%%%%%%%%%%%%%%%%%%%%%%%%%%%%%%%%%%%%%%%%%%%%%%%%%%%%%%%%%%%%%%%%%%
%                              Document Settings                               %
%%%%%%%%%%%%%%%%%%%%%%%%%%%%%%%%%%%%%%%%%%%%%%%%%%%%%%%%%%%%%%%%%%%%%%%%%%%%%%%%

\documentclass[a4paper,12pt]{report}

%%%%%%%%%%%%%%%%%%%%%%%%%%%%%%%%%%%%%%%%%%%%%%%%%%%%%%%%%%%%%%%%%%%%%%%%%%%%%%%%
%                                   Margins                                    %
%%%%%%%%%%%%%%%%%%%%%%%%%%%%%%%%%%%%%%%%%%%%%%%%%%%%%%%%%%%%%%%%%%%%%%%%%%%%%%%%

\addtolength{\oddsidemargin}{-1.cm}
\addtolength{\textwidth}{2cm}
\addtolength{\topmargin}{-2cm}
\addtolength{\textheight}{3.5cm}

%%%%%%%%%%%%%%%%%%%%%%%%%%%%%%%%%%%%%%%%%%%%%%%%%%%%%%%%%%%%%%%%%%%%%%%%%%%%%%%%
%                                   Packages                                   %
%%%%%%%%%%%%%%%%%%%%%%%%%%%%%%%%%%%%%%%%%%%%%%%%%%%%%%%%%%%%%%%%%%%%%%%%%%%%%%%%



\usepackage[pdftex]{graphicx}	% Include graphics
\usepackage{float}				% Place floats inline, use the [H] placement
\usepackage{array}				% Additional tables features
\usepackage[utf8]{inputenc}
\usepackage{amsmath}
\usepackage{graphicx}
\usepackage{float}
\usepackage{listings}
\usepackage{pdfpages}
\usepackage{enumerate}
\usepackage{amsmath}
\usepackage{amsfonts}
\usepackage{amssymb}
\usepackage[mathscr]{euscript}
\usepackage[explicit]{titlesec}
\usepackage{times}				% Times new roman font

%%%%%%%%%%%%%%%%%%%%%%%%%%%%%%%%%%%%%%%%%%%%%%%%%%%%%%%%%%%%%%%%%%%%%%%%%%%%%%%%
%								Commands 									   %
%%%%%%%%%%%%%%%%%%%%%%%%%%%%%%%%%%%%%%%%%%%%%%%%%%%%%%%%%%%%%%%%%%%%%%%%%%%%%%%%

\renewcommand\thesection{\arabic{section}} 
\titleformat{\section}{\normalfont\fontsize{16}{15}\bfseries\uppercase}{\thesection}{1em}{\underline{#1}}
\titleformat{\subsection}{\normalfont\fontsize{14}{15}\bfseries\uppercase}{\thesubsection}{1em}{#1}
\titleformat{\subsubsection}{\normalfont\fontsize{12}{15}\bfseries\itshape}{\thesubsubsection}{1em}{#1}

%%%%%%%%%%%%%%%%%%%%%%%%%%%%%%%%%%%%%%%%%%%%%%%%%%%%%%%%%%%%%%%%%%%%%%%%%%%%%%%%
%                                   Document                                   %
%%%%%%%%%%%%%%%%%%%%%%%%%%%%%%%%%%%%%%%%%%%%%%%%%%%%%%%%%%%%%%%%%%%%%%%%%%%%%%%%
\begin{document}

	% Title page
	%\input{title_page.tex}


\pagebreak
\tableofcontents
\pagebreak

\setcounter{page}{1}

\section{Introduction}
This is where you put the work into perspective and state what the practical is about. Answer the following questions:
\begin{itemize}
\item What is the origin or background?
\item What is the aim? What problem will be addressed?
\item What had to be done?
\item What are the expected outcomes?
\end{itemize}
Do not copy the introduction from the study guide. Keep the introduction concise.

\section{Theoretical Background}
Briefly give the theory underlying your practical. It can include simplified circuit diagrams, theory, assumptions, mathematics required for data evaluation and interpretation. Keep it brief as this is not the main part of your practical assignment.
\section{Design $\backslash$ Method}
Give all the details of your design or experimental procedure.
If you had to design a circuit, include the following:
\begin{itemize}
\item Design equations
\item Assumptions
\item Final circuit diagram. If your circuit consists of different stages, give a block diagram and the circuit for each block.
\end{itemize}

If you have to take measurements, include the following:
\begin{itemize}
\item What you had to measure
\item Equipment used
\item Procedure – when did you measure what and how? What were the steps you followed?
\end{itemize}
Giving circuits from the internet of how your circuit can look is not a design, you have to present your own circuit, i.e. the one that you designed from first principles. You may not (ever!) paste circuits that you copied from the Internet into any report as a schematic of a circuit that you designed.
Do not put photos of your design from your labbook in the report. It is not neat at all and is in any case forbidden in the reports of many modules' including those of Project EPR400.
\section{Simulations}

Include any simulations you did to test your design or circuit. These can be in Matlab/Octave/Python, OrCAD/LTSpice or any other simulation package that can simulate your design effectively. Do not include the scripts unless there is a novel simulation technique that you need to explain. Explain the simulation setup, i.e. which software did you use, how did you set up the simulation, what were the parameters and boundary conditions? Then provide the simulation results (also called predictions) in the form of graphs or tables and an explanation thereof. The explanation should point out the characteristics of the simulation results that are important to the task, e.g. the bandwidth of a filter or the phase shift between two points in a circuit.

\section{Results}

The following should be in this section:

\begin{itemize}
\item Any measurements you took, e.g. voltage, current, frequency, noise etc.
\item Graphs and/or tables of measurements
\item Pictures of the measurements on the oscilloscope (e.g. waveform, frequency spectrum, controller output). Do not include photos of the oscilloscope taken with your phone, rather export the oscilloscope data on a memory stick via the USB and use that. Photos of a signal on an oscilloscope screen are forbidden in project EPR400 reports, so you may as well work out how to export the data from the oscilloscope sooner rather than later to enable you to prepare a professional-looking report.
\item If you need to repeat a measurement a number of times, e.g. to determine how accurate1 a sensor detects an object, you need to give a rudimentary statistical analysis. For example, if the sensor detects an object 5 out of 10 times, its accuracy is 50%. It would not make sense to take only two measurements in this case, so make sure that your sample number is large enough to reflect the true statistical properties of the system.
\end{itemize}
A photo of your circuit on protoboard is not a result. A result is anything you can measure. 
\section{Discussion}

Discuss your results. What do the results mean? Compare the measured results with your simulations and with the desired outcome. Did you meet the specifications? If you did not meet the specifications or desired outcome, why not? What can you change in your design to improve it? If you had to consult the literature, how do your observations compare to those of other people?

\section{Conclusion}
A summary of what has been done, the primary results obtained and observations made.  This section must link with the original, primary objectives, i.e. one should be able to read the introduction and the conclusion and be able to grasp the essence of the report.  

\section{References}

...

\end{document}